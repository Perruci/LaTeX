\documentclass[hyperref={pdfpagelabels=false}]{beamer}
\usepackage{lmodern}
\usepackage[T1]{fontenc}
\usepackage[english]{babel}
\usepackage{lara}
\usepackage{epstopdf}



%%------------------- Front Cover -------------------%% 
\title{Presentation Long Title}
\institute{E-mail\\ \vspace{15pt}Institution Name}
\author{Author Name}
\date{Date}


%%--------------[Optional] Headline Options -------------------%% 
%% Manual adjust the headline size according to the number of sections you would like
%% to appear in the headline. Useful to eliminate the references from the headline.
% \headLineMAXSections{2}


%%--------------[Optional] Footline Options -------------------%% 
%% If you like to use the Tradional footline, i.e., [Left: Full name], [right: Full title] ==> Comment all lines below.
%%
%%%%%%%%%% Other options (1) - Right and Left footline (no changes in the division width):
\manualFootlineTitle{Authors in footline - right}
\manualFootlineAuthor{Authors in footline - left}
%%
%%%%%%%%%% Other options (2) - only Authors name (i.e. width 97 = maximum):
% \manualFootlineTitle{}
% \manualFootlineAuthor{Authors in the footline}
% \manualFootlineWidth{97}
%%
%%%%%%%%%% Other options (3) - only title (i.e. width 0 = minimum):
% \manualFootlineTitle{Title in footline}
% \manualFootlineAuthor{}
% \manualFootlineWidth{0}




%%------------------- BEGIN DOCUMENT-------------------%% 
\begin{document}
\maketitle
\makesummary{Table of Contents}



%%--------------- Presentation Content ---------------%%

\section{Section 1}
\subsection{Subsection 1}
\frame
{
  \frametitle{Slide Title}
  \vspace{25pt}
    \begin{enumerate}
     \item First item  
     \begin{itemize}
      \item First subitem
      \item Second subitem
     \end{itemize}
     \item Second item
     \item Third item
    \end{enumerate}
}
\frame
{
  \frametitle{Slide with overlay}
  \vspace{25pt}
    \begin{enumerate}
     \item First item \pause
     \begin{itemize}
      \item First subitem
      \item Second subitem
     \end{itemize}
     \item Second item \pause
     \item Third item
    \end{enumerate}
}
\subsection{Subsection 2}
\frame 
{
  \frametitle{Slide with columns and blocks}
  \vspace{25pt}
  \begin{columns}[c]
    \column{0.5\textwidth}
    \begin{block}{Block title}
      \begin{itemize}
	\item Item 1
	\item Item 2
	\end{itemize}
    \end{block}
    %Block Branco:
    \blockbranco{Title in blue}{ body in black}
    \column{0.5\textwidth}
      \includegraphics[width=0.5\textwidth]{LARA_logo.eps}
   \end{columns}
}
\frame 
{
  \frametitle{Slide with figure in overlay}
  \begin{columns}[c]
    \column{0.5\textwidth}
      \includecoveredgraphics[width=0.5\textwidth]{LARA_logo.eps} \pause
    \column{0.5\textwidth}
      \includecoveredgraphics[width=0.5\textwidth]{LARA_logo.eps}
   \end{columns}
}
\section{Section 2}
\subsection{Another Subsection}
\frame
{
  \frametitle{Animated images}
  \begin{beamerboxesrounded}[lower=date in head/foot,shadow=true]{}
    This is a beamer box... \cite{article:dummy}
    Dummy text, dummy text, dummy text, dummy text, dummy text, dummy text, 
    dummy text, dummy text, dummy text, dummy text, dummy text, dummy text, 
    dummy text, dummy text, dummy text, dummy text, dummy text, dummy text, 
    dummy text, dummy text, dummy text, dummy text, dummy text, dummy text.
  \end{beamerboxesrounded}

    To use animation, make a sequence of images with same name in number sequence.
    \setbeamercovered{invisible}
    \begin{center}
      %\multiinclude[format=eps,graphics={width=\textwidth}]{fig} %fig-0.eps, fig-1.eps, ..., fig-n.eps
    \end{center} 
    \setbeamercovered{transparent}
}
\frame
{
  \frametitle{Tables}
  \begin{table}
   \begin{tabular}{c|c|c|c|}
      \cline{2-4}& \multicolumn{3}{c|}{xxx}\\ \hline
      \multicolumn{1}{|c|}{A} & B & C & D \\ \hline
      \multicolumn{1}{|c|}{A1} & & & \\ \hline
      \multicolumn{1}{|c|}{A2} & & & \\ \hline
      \multicolumn{1}{|c|}{A3} & & & \\ \hline
      \multicolumn{1}{|c|}{A4} & & & \\ \hline
      \multicolumn{1}{|c|}{A5} & & & \\ \hline
    \end{tabular}
  \end{table}
}
\tiny\bibliography{bibliography}

\end{document}