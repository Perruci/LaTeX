%TCIDATA{LaTeXparent=0,0,relatorio.tex}
                      
\chapter{Introdu��o}\label{CapIntro}

% Resumo opcional. Comentar se n�o usar.
\resumodocapitulo{Este cap�tulo apresenta a principal motiva��o do trabalho de gradua��o. Os objetivos s�o claramente apresentados, visando assim satisfazer um conjunto de caracter�sticas prescritas para este trabalho. Por fim, o manuscrito � apresentado. (Este resumo � opcional)}

\section{Contextualiza��o}

Em geral, na introdu��o que � feita a contextualiza��o do trabalho. Aspectos importantes tais como motiva��o e relev�ncia do tema escolhido devem ficar claros.

\section{Defini��o do problema}

Aqui o problema � definido.

\section{Objetivos do projeto}

Nesta se��o, deve-se deixar claro quais s�o os objetivos do projeto.

\section{Apresenta��o do manuscrito}

No cap�tulo \ref{CapRevisaoBibliografica} � feita uma revis�o bibliogr�fica sobre o tema de estudo. Em seguida, o cap�tulo \ref{CapDesenvolvimento} descreve a metodologia empregada no desenvolvimento do projeto. Resultados experimentais s�o discutidos no cap�tulo \ref{CapExperimentos}, seguido das conclus�es no cap�tulo \ref{CapConclusoes}. Os anexos cont�m material complementar.

 

